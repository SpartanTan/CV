% Resume in Latex
% Author : Jake Gutierrez
% Based off of: https://github.com/sb2nov/resume
% License : MIT
% Adjusted for Chinese Support and XeLaTeX compilation

\documentclass[letterpaper,11pt]{article}

\usepackage{latexsym}
\usepackage[empty]{fullpage}
\usepackage{titlesec}
\usepackage{marvosym}
\usepackage[usenames,dvipsnames]{color}
\usepackage{verbatim}
\usepackage{enumitem}
\usepackage[hidelinks]{hyperref}
\usepackage{fancyhdr}
\usepackage{tabularx}
\usepackage{xeCJK} % Support for Chinese characters
\setmainfont{Arial}  % 将主字体设置为 Arial
\setCJKmainfont[BoldFont=Microsoft YaHei]{SimSun} % Set the main font to SimSun, or any other preferred font
\usepackage{fontawesome}  % 加载 FontAwesome 图标库
\usepackage{multirow}

\pagestyle{fancy}
\fancyhf{} % clear all header and footer fields
\fancyfoot{}
\renewcommand{\headrulewidth}{0pt}
\renewcommand{\footrulewidth}{0pt}

% Adjust margins
\addtolength{\oddsidemargin}{-0.5in}
\addtolength{\evensidemargin}{-0.5in}
\addtolength{\textwidth}{1in}
\addtolength{\topmargin}{-.5in}
\addtolength{\textheight}{1.0in}

\urlstyle{same}

\raggedbottom
\raggedright
\setlength{\tabcolsep}{0in}

% Sections formatting
\titleformat{\section}{
  \vspace{-4pt}\scshape\raggedright\large\bfseries
}{}{0em}{}[\color{black}\titlerule \vspace{-5pt}]

% Custom commands
\newcommand{\resumeItem}[1]{
  \item\small{
    {#1 \vspace{-2pt}}
  }
}

\newcommand{\resumeSubheading}[4]{
  \vspace{-2pt}\item\small{
    \begin{tabular*}{0.97\textwidth}[t]{l@{\extracolsep{\fill}}r}
      \textbf{#1} & \textbf{#2} \\
      \textit{#3} & \textit{#4} \\
    \end{tabular*}\vspace{-7pt}
  }
}

% 新命令,用于需要额外行的条目
\newcommand{\resumeSubheadingExtended}[5]{  
  \item\small{
    \begin{tabular*}{0.97\textwidth}{@{\extracolsep{\fill}}lr}
      \textbf{#1} & \textbf{#2} \\  % 第一行:学校名称 & 地点
      \textit{#3} & \textit{#4} \\  % 第二行:专业名称 & 时间
      #5 &  % 第三行:GPA,仅左侧
    \end{tabular*}\vspace{-10pt}
  }
}


\newcommand{\resumeSubItem}[1]{\resumeItem{#1}\vspace{-4pt}}

\newcommand{\resumeSubHeadingListStart}{\begin{itemize}[leftmargin=0.15in, label={}]}
\newcommand{\resumeSubHeadingListEnd}{\end{itemize}}
\newcommand{\resumeItemListStart}{\begin{itemize}}
\newcommand{\resumeItemListEnd}{\end{itemize}\vspace{-5pt}}

\begin{document}

% \begin{center}
%     \textbf{\Huge \scshape 谈至存} \\ % Chinese Name
%     \vspace{5pt}
%     \textbf{生日:} 1998.06.06 \ \textbf{居住地:} 江苏苏州  \ \textbf{电话:} +(86) 13402536577 \\
%     \vspace{5pt}
%     \textbf{邮箱: }\href{mailto:tanzc9866@126.com}{tanzc9866@126.com} \ \href{https://github.com/SpartanTan}{\faGithub Github}\\ % 邮件
% \end{center}

\vspace*{-10mm}
\noindent
\begin{tabular}{@{}l@{}r@{\hspace{10cm}}l@{}}
    \multirow{3}[6]{*}{\textbf{\Huge 谈至存}} & & \begin{tabular}[t]{@{}l@{}}
        \textbf{生日:} 1998.06.06 \\
        \textbf{居住地:} \textbf{江苏苏州} \\
        \textbf{电话:} +(86) 13402536577 \\
        \textbf{邮箱:} \href{mailto:tanzc9866@126.com}{tanzc9866@126.com} \\
        \href{https://github.com/SpartanTan}{\faGithub Github}
    \end{tabular} \\
\end{tabular}

%--- Insert sections of resume here ---
\vspace{-5mm}
\section{个人自述}
我曾在丹麦科技大学和查尔姆斯理工大学学习自主系统,参与过赛车动力学仿真、机器人导航控制等项目的软件开发。在赛车队和实验室的经历使我具有坚实的问题分析能力、实践能力、较强的团队合作能力和沟通能力。

% \vspace{-3mm}
\vspace{-3mm}
%-----------EDUCATION-----------
\section{教育经历}
  \resumeSubHeadingListStart
    \resumeSubheadingExtended
      {丹麦科技大学(Technical University of Denmark)}{哥本哈根,丹麦}
      {硕士学位,\textbf{自治系统(Autonomous Systems)}}{2021.01 -- 2023.12}  
      {GPA: 8.86/12; 核心课程:线性系统设计、自治系统的感知、自主机器人}
  \resumeSubHeadingListEnd

  \vspace{0.5mm} % 调整这里的值以减少空间

  \resumeSubHeadingListStart
    \resumeSubheadingExtended
      {查尔姆斯理工大学(Chalmers University of Technology)- 交换生}{哥德堡,瑞典}
      {\textbf{系统控制与机电一体化(Systems, Control and Mechatronics)}}{2022.09 -- 2023.12}
      {核心课程:车辆动力学工程、自治系统的决策系统、人工神经网络}
  \resumeSubHeadingListEnd
  
  \vspace{0.5mm} % 调整这里的值以减少空间

  \resumeSubHeadingListStart
    \resumeSubheadingExtended
      {南京工程学院 - 本科}{南京,江苏}
      {学士学位,\textbf{机械电子工程}}{2016.09 -- 2020.06}
      {GPA: 3.47; 专业排名 10/162}
  \resumeSubHeadingListEnd

%-----------Extracurriculars-----------
\section{项目}
  \resumeSubHeadingListStart

    \resumeSubheading
      {\textbf{强化学习机器人导航控制 \; \href{https://1drv.ms/v/s!Al-YZOpjHxorgslaTIu4oeTj8TYROw?e=EflSJj}{\faVideoCamera} \;  \href{https://github.com/SpartanTan/RLATR}{\faGithub}}}{丹麦科技大学}
      {独立开发者}{2023.07 -- 2023.12}
      \resumeItemListStart
        \resumeItem{基于Gymnasium环境开发了一个搭载激光雷达两轮驱动小车的训练环境}
        \resumeItem{由于环境限制,从零实现了激光雷达模块,尝试采用多种方式如numba和向量化加速计算}
        \resumeItem{为避免过拟合,实现了可随机生成的训练环境,包含可调长度的规划路径、可动障碍物、及走廊墙壁}
        \resumeItem{开发了一个用于调试训练参数、环境参数和图形化显示仿真环境的GUI工具,便于显示数据和验证算法}
        \resumeItem{实现了基于PPO的导航控制算法,使得小车能够在未知环境中避障并到达目标,综合成功率在60\%左右}
    \resumeItemListEnd
      
    \resumeSubheading
      {\textbf{Unmanned autonomous systems 无人自主系统  \; \href{https://1drv.ms/v/s!Al-YZOpjHxorgslZm5SJR7mVF_eoPw?e=4kRkvd}{\faVideoCamera} \;  \href{https://github.com/SpartanTan/31390-UAS-2022}{\faGithub}}}{丹麦科技大学}
      {项目开发者}{2022.06 -- 2022.06}
      \resumeItemListStart
        \resumeItem{实现基于Matlab/Simulink的无人机控制系统,使得小型四轴无人机实现悬停、平移、定点飞行等功能}
        \resumeItem{实现了3D环境下应用A*进行路径规划,成功控制无人机穿越3D迷宫}
        \resumeItem{应用多项式优化工具实现轨迹规划,成功控制无人机自主起飞、穿越四个随机设置的圆环并降落}
      \resumeItemListEnd
  \resumeSubHeadingListEnd


%-----------Experience-----------
\section{实习经历}
  \resumeSubHeadingListStart
    \resumeSubheading
    {\href{https://www.chalmersformulastudent.se/cfs-2023-car-margareta}{Chalmers Formula Student/Chalmers方程式赛车队 \faExternalLink}}{查尔姆斯理工大学}
      {自动驾驶团队软件工程师}{2022.09 -- 2023.08}
      \resumeItemListStart
        \resumeItem{实现从GPS获取的地理坐标(经纬度)转换为地图坐标的方法。该方法被用以验证SLAM地图的准确性。}
        \resumeItem{开发Gazebo插件,实现以扭矩为输入、考虑重量转移的四轮车辆动力学仿真,便于测试SLAM和控制算法}
        \resumeItem{参与开发基于PyQt的GUI工具,便于启动Gazebo、Rviz以及自动驾驶系统,可根据实际车辆调整模型参数}
        \resumeItem{帮助团队在2023 Formula Student德国站无人杯获得\textbf{冠军},东欧站无人杯获得第七名}
    \resumeItemListEnd
  %   \resumeSubheading
    %   {凉善公益}{龙门乡塔哈村小学, 凉山彝族自治州}
    %   {\small 小学支教老师}{2024.02 -- 2024.06}
    %   \resumeItemListStart
    %     \resumeItem{二年级语文老师,班主任;三、四年级英语老师}
    %     \resumeItem{期末成绩全县排名17,同类学校排名第5}
  %   \resumeItemListEnd

    % \resumeSubheading
    %   {苏州盖茨电子科技有限公司}{苏州,江苏}
    %   {电子研发实习生}{2020.09 -- 2020.10}
    %   \resumeItemListStart
    %     \resumeItem{参与调试基于S32K144的车载空压机FOC算法}
    %   \resumeItemListEnd
      
    \resumeSubheading
      {越野机器人实验室}{南京工程学院,南京}
      {单片机开发、负责人}{2017.07 -- 2018.07}
      \resumeItemListStart
        \resumeItem{开发基于STM32的四轮循迹小车程序,实现摄像头/激光测距循迹,机械臂控制}
        \resumeItem{带领团队蝉联2017年省大学生机器人大赛和2018年中国工程机器人大赛冠军}
        \resumeItem{申请2项实用新型专利,完成1项挑战杯(管道机器人)结题}
        \resumeItemListEnd

  \resumeSubHeadingListEnd


\section{\href{https://1drv.ms/f/s!Al-YZOpjHxorgslbx6yg1s1e4ZGl1A?e=1DVmBG}{奖项 \faExternalLink}}
    \resumeSubHeadingListStart
      \resumeSubheading
      {2019钛马大赛 冠军 -(室内环境下搭载单线激光雷达阿克曼转向小车避障越野)}{2019.06}
      {}{}
      \vspace{-4mm}

      \resumeSubheading
      {2018中国工程机器人大赛 冠军 -(工程越野项目竞技赛:摄像头循迹、PID控制、色彩识别、机械臂抓取)}{2018.04}
      {}{}
      \vspace{-4mm}

      % \resumeSubheading
      % {第八届江苏省机器人大赛 冠军 -(机器人越野项目:摄像头循迹、PID控制)}{2017.11}
      % {}{}
      % \vspace{-4mm}

      \resumeSubheading
      {2017中国机器人大赛 一等奖 -(机器人越野项目:摄像头循迹、PID控制)}{2017.08}
      {}{}
      \vspace{-4mm}
      % \resumeSubheading
      %   {National Junior Honor Society}{June 2023 -- Present}
      %   {}{}
      %   \resumeItemListStart
      %   \vspace{-4mm}
      %     \resumeItem{Award presented to students with outstanding achievements in academics, leadership, and community service.}
      %   \resumeItemListEnd
    \resumeSubHeadingListEnd

%-----------Courses-----------
% \section{Classes}
%  \begin{itemize}[leftmargin=0.15in, label={}]
%     \small{\item{
%     \vspace{1mm}
%      \textbf{Courses}{: AP Calulus BC, AP Chemistry, AP Physics, AP World History, AP U.S. History, AP Statistics, AP Literature} \\

%     }}
%  \end{itemize}

%-----------Skill and Interests-----------
\section{技能能力}
 \begin{itemize}[leftmargin=0.15in, label={}]
    \small{\item{
    \vspace{1mm}
     \textbf{编程相关}{: Python、C/C++、Matlab/Simulink、Linux、 ROS、Docker、git} \\
     \vspace{1mm}
     \textbf{专业相关}{: 线性控制系统搭建和分析、PID、MPC等控制方法、强化学习和控制、无人机/车辆/水下机器人动力学及控制} \\
     \vspace{1mm}
     \textbf{语言相关}{:英语雅思6.5(口语7.0),曾深度参与外国工程团队,具有较强沟通能力}

    }}
 \end{itemize}

%-----------Contact-----------
%  \begin{center}
% \small 123-456-7890 $|$
% \href{mailto:ADD EMAIL HERE@x.com}{\underline{andrew@gmail.com}} $|$
% \href{ADD LINKEDIN PAGE HERE}{\underline{linkedin.com/in/andrew}} 
% \end{center}

\end{document}
