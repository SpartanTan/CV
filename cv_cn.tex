% Resume in Latex
% Author : Jake Gutierrez
% Based off of: https://github.com/sb2nov/resume
% License : MIT
% Adjusted for Chinese Support and XeLaTeX compilation

\documentclass[letterpaper,11pt]{article}

\usepackage{latexsym}
\usepackage[empty]{fullpage}
\usepackage{titlesec}
\usepackage{marvosym}
\usepackage[usenames,dvipsnames]{color}
\usepackage{verbatim}
\usepackage{enumitem}
\usepackage[hidelinks]{hyperref}
\usepackage{fancyhdr}
\usepackage{tabularx}
\usepackage{xeCJK} % Support for Chinese characters
\setmainfont{Arial}  % 将主字体设置为 Arial
\setCJKmainfont[BoldFont=Microsoft YaHei]{SimSun} % Set the main font to SimSun, or any other preferred font
\usepackage{fontawesome}  % 加载 FontAwesome 图标库
\usepackage{multirow}

\pagestyle{fancy}
\fancyhf{} % clear all header and footer fields
\fancyfoot{}
\renewcommand{\headrulewidth}{0pt}
\renewcommand{\footrulewidth}{0pt}

% Adjust margins
\addtolength{\oddsidemargin}{-0.5in}
\addtolength{\evensidemargin}{-0.5in}
\addtolength{\textwidth}{1in}
\addtolength{\topmargin}{-.5in}
\addtolength{\textheight}{1.0in}

\urlstyle{same}

\raggedbottom
\raggedright
\setlength{\tabcolsep}{0in}

% Sections formatting
\titleformat{\section}{
  \vspace{-4pt}\scshape\raggedright\large\bfseries
}{}{0em}{}[\color{black}\titlerule \vspace{-5pt}]

% Custom commands
\newcommand{\resumeItem}[1]{
  \item\small{
    {#1 \vspace{-2pt}}
  }
}

\newcommand{\resumeSubheading}[4]{
  \vspace{-2pt}\item\small{
    \begin{tabular*}{0.97\textwidth}[t]{l@{\extracolsep{\fill}}r}
      \textbf{#1} & \textbf{#2} \\
      \textit{#3} & \textit{#4} \\
    \end{tabular*}\vspace{-7pt}
  }
}

% 新命令,用于需要额外行的条目
\newcommand{\resumeSubheadingExtended}[5]{  
  \item\small{
    \begin{tabular*}{0.97\textwidth}{@{\extracolsep{\fill}}lr}
      \textbf{#1} & \textbf{#2} \\  % 第一行:学校名称 & 地点
      \textit{#3} & \textit{#4} \\  % 第二行:专业名称 & 时间
      #5 &  % 第三行:GPA,仅左侧
    \end{tabular*}\vspace{-10pt}
  }
}


\newcommand{\resumeSubItem}[1]{\resumeItem{#1}\vspace{-4pt}}

\newcommand{\resumeSubHeadingListStart}{\begin{itemize}[leftmargin=0.15in, label={}]}
\newcommand{\resumeSubHeadingListEnd}{\end{itemize}}
\newcommand{\resumeItemListStart}{\begin{itemize}}
\newcommand{\resumeItemListEnd}{\end{itemize}\vspace{-5pt}}

\begin{document}

% \begin{center}
%     \textbf{\Huge \scshape 谈至存} \\ % Chinese Name
%     \vspace{5pt}
%     \textbf{生日:} 1998.06.06 \ \textbf{居住地:} 江苏苏州  \ \textbf{电话:} +(86) 13402536577 \\
%     \vspace{5pt}
%     \textbf{邮箱: }\href{mailto:tanzc9866@126.com}{tanzc9866@126.com} \ \href{https://github.com/SpartanTan}{\faGithub Github}\\ % 邮件
% \end{center}

\vspace*{-10mm}
\noindent
\begin{tabular}{@{}l@{}r@{\hspace{10cm}}l@{}}
    \multirow{3}[6]{*}{\textbf{\Huge 谈至存}} & & \begin{tabular}[t]{@{}l@{}}
        \textbf{生日:} 1998.06.06 \\
        \textbf{居住地:} \textbf{江苏苏州} \\
        \textbf{电话:} +(86) 13402536577 \\
        \textbf{邮箱:} \href{mailto:tanzc9866@126.com}{tanzc9866@126.com} \\
        %\href{https://github.com/SpartanTan}{\faGithub Github}
        \href{https://github.com/SpartanTan}{\faGithub} \ \href{www.linkedin.com/in/tanzhicun}{\faLinkedin}
    \end{tabular} \\
\end{tabular}

%--- Insert sections of resume here ---
\vspace{-5mm}
\section{个人自述}
我获得丹麦科技大学自主系统的硕士学位,曾参与开发无人赛车状态估计器、控制器和动力学仿真;无人机规划控制以及深度强化学习控制器的项目。在赛车队和实验室的经历使我在工作中能积极地参与沟通,分析问题,快速实践。

% \vspace{-3mm}
\vspace{-3mm}
%-----------EDUCATION-----------
\section{教育经历}
  \resumeSubHeadingListStart
    \resumeSubheadingExtended
      {丹麦科技大学(Technical University of Denmark)}{哥本哈根,丹麦}
      {硕士学位,\textbf{自主系统(Autonomous Systems)}}{2021.01 -- 2023.12}  
      {GPA: 8.86/12; 课程:线性系统设计、自治系统的感知、基于模型的系统工程,强化学习与控制概论}
  \resumeSubHeadingListEnd

  \vspace{0.5mm} % 调整这里的值以减少空间

  \resumeSubHeadingListStart
    \resumeSubheadingExtended
      {查尔姆斯理工大学(Chalmers University of Technology)- 交换生}{哥德堡,瑞典}
      {\textbf{系统控制与机电一体化(Systems, Control and Mechatronics)}}{2022.09 -- 2023.12}
      {课程:车辆动力学工程、建模与仿真、自治系统的决策系统、人工神经网络、模型预测控制}
  \resumeSubHeadingListEnd
  
  \vspace{0.5mm} % 调整这里的值以减少空间

  \resumeSubHeadingListStart
    \resumeSubheadingExtended
      {南京工程学院 - 本科}{南京,江苏}
      {学士学位,\textbf{机械电子工程}}{2016.09 -- 2020.06}
      {GPA: 3.47; 专业排名 10/162}
  \resumeSubHeadingListEnd

%-----------Experience-----------
\section{工作经历}
\resumeSubHeadingListStart
\resumeSubheading
{畅加风行(苏州)智能科技有限公司}{苏州,江苏}
{规划控制工程师}{2024.12 -- 至今}
\resumeItemListStart
  \resumeItem{以C++重写开源预测算法Simpl的数据预处理部分,耗时降低至10ms以内}
  \resumeItem{参考apollo代码开发回溯碰撞检测功能,将开源GJK算法集成进PNC模块中,综合快20\%}
  \resumeItem{掌握车辆横纵向标定,相机及雷达内外参标定方法,输出多份文档}
  % \resumeItem{打通从手持建图设备到道路拓扑地图转换流程}
  \resumeItem{参与规控模块实车部署和调试,包括乘用车、无人快递车和越野车等}
  \resumeItem{重构控制部分代码,整理、封装}
  \resumeItemListEnd 

\resumeSubHeadingListEnd

%-----------Experience-----------
\section{实习经历}
\resumeSubHeadingListStart
    \resumeSubheading
    {宏景智驾-无人赛车“天猿”自动驾驶系统项目}{株洲国际赛车场}
    {\small 创新研发部实习生(Matlab,车辆横向控制,车辆动力学仿真,速度规划器)}{2024.10}
    \resumeItemListStart
      % \resumeItem{参与开发基于惯导RTK定位的无人驾驶汽车横向控制实车调试及仿真}
      % \resumeItem{负责建立车辆动力学模型,设计并联PID控制航向误差和横向轨迹误差,在仿真中验证可行性。最终 \\ 取得横向误差RMS小于2}
      \resumeItem{实现速度规划器。使用Menger方法计算赛道曲率,并根据最大横向加速度计算路段最大速度}
      \resumeItem{实现前向/后向调整的方法,确保速度规划符合车辆最大加/减速度限制,避免进弯前急减速和出弯后的急加速}
      \resumeItem{参与实车控制系统参数调试,所调车辆在株洲赛车场取得2分40秒的成绩}
    \resumeItemListEnd  
  
    \resumeSubheading
  {\href{https://www.chalmersformulastudent.se/cfs-2023-car-margareta}{Chalmers Formula Student/Chalmers方程式赛车队 \faExternalLink}}{查尔姆斯理工大学}
    {自动驾驶团队软件工程师 (C++,Gazebo,ROS,卡尔曼滤波器,车辆动力学)}{2022.09 -- 2023.08}
    \resumeItemListStart
      % \resumeItem{参与开发和维护包含感知、规划、控制、通信等系统的大型自动驾驶项目}
      \resumeItem{负责移植eufs\_sim到ROS2和新版Gazebo中,解决API冲突问题}
      \resumeItem{负责开发基于动力学的四轮驱动车辆运动仿真插件,增加横/纵向重量转移计算以及使用 \\ RK4提高仿真精确度}
      \resumeItem{参与开发了基于PyQt的自动驾驶系统测试软件,开发一键启动,参数选择等功能,打包含有整套 \\ 仿真软件的docker环境}
      \resumeItem{参与赛车状态估计器的开发。基于LuGre轮胎模型,实现含有15个状态的卡尔曼滤波器。通过 \\ 轮速计和IMU精确估计车速与角速度,为感知和控制提供准确估计信息}
      \resumeItem{帮助团队在2023 Formula Student德国站无人杯获得\textbf{冠军},东欧站无人杯获得第七名}
  \resumeItemListEnd
%   \resumeSubheading
  %   {凉善公益}{龙门乡塔哈村小学, 凉山彝族自治州}
  %   {\small 小学支教老师}{2024.02 -- 2024.06}
  %   \resumeItemListStart
  %     \resumeItem{二年级语文老师,班主任;三、四年级英语老师}
  %     \resumeItem{期末成绩全县排名17,同类学校排名第5}
%   \resumeItemListEnd

  % \resumeSubheading
  %   {苏州盖茨电子科技有限公司}{苏州,江苏}
  %   {电子研发实习生}{2020.09 -- 2020.10}
  %   \resumeItemListStart
  %     \resumeItem{参与调试基于S32K144的车载空压机FOC算法}
  %   \resumeItemListEnd

  % \resumeSubheading
  %   {邦纳电子(苏州)有限公司}{苏州,江苏}
  %   {电子研发实习生}{2019.07 -- 2019.08}
  %   \resumeItemListStart
  %     \resumeItem{使用Oracle系统建立电子零件库}
  %     \resumeItem{参与了塔灯和光电传感器项目,通过更换元件对比亮度和功耗等性能参数}
  %     \resumeItem{使用示波器、分贝计等工具测量并分析器件增益、迟滞性和波束}
  %   \resumeItemListEnd
    
  \resumeSubheading
    {越野机器人实验室}{南京工程学院,南京}
    {单片机开发、负责人(嵌入式软件,C,PID,团队管理)}{2017.07 -- 2018.07}
    \resumeItemListStart
      \resumeItem{开发基于STM32的四轮循迹小车程序,实现摄像头/激光测距循迹,机械臂控制}
      \resumeItem{带领团队蝉联2017年省大学生机器人大赛和2018年中国工程机器人大赛冠军}
      \resumeItem{申请2项实用新型专利,完成1项挑战杯(管道机器人)结题}
      \resumeItemListEnd

\resumeSubHeadingListEnd

%-----------Extracurriculars-----------
\section{项目}
  \resumeSubHeadingListStart

    \resumeSubheading
      {\textbf{强化学习机器人导航控制 \; \href{https://1drv.ms/v/s!Al-YZOpjHxorgslaTIu4oeTj8TYROw?e=EflSJj}{\faVideoCamera} \;  \href{https://github.com/SpartanTan/RLATR}{\faGithub}}}{丹麦科技大学}
      {独立开发者(机器人运动学仿真,Python,强化学习,PyQt GUI设计)}{2023.07 -- 2023.12}
      \resumeItemListStart
        \resumeItem{基于Gymnasium环境开发了一个两轮驱动小车的运动学模型和激光雷达仿真模块}
        \resumeItem{将工厂地图转换为Polygon,实现符合实机测试需要的拟真环境}
        \resumeItem{开发算法,实现随机生成路径、可动障碍物和走廊墙壁。提供更具普适性和挑战性的训练环境}
        \resumeItem{开发了一个用于调试训练参数、环境参数和图形化显示仿真环境的GUI工具,便于显示数据和验证算法}
        \resumeItem{根据任务需要,设计并实现路径跟踪和避障的奖励函数,并测试不同权重下的性能表现}
        \resumeItem{实现PPO算法训练神经网络控制器,在未知环境中导航避障成功率达到60\%}
        % \resumeItem{测试DDPG、A3C等算法在此环境中的表现,与PPO算}
    \resumeItemListEnd
      
    \resumeSubheading
      {\textbf{Unmanned autonomous systems 无人自主系统  \; \href{https://1drv.ms/v/s!Al-YZOpjHxorgslZm5SJR7mVF_eoPw?e=4kRkvd}{\faVideoCamera} \;  \href{https://github.com/SpartanTan/31390-UAS-2022}{\faGithub}}}{丹麦科技大学}
      {项目开发者(Matlab/Simulink,无人机动力学建模仿真,A*路径规划)}{2022.06 -- 2022.06}
      \resumeItemListStart
        \resumeItem{推导动力学模型,在Simulink中构建非线性模型并线性化获得状态空间模型}
        \resumeItem{负责实现基于Matlab/Simulink的PID控制器,控制无人机实现悬停、平移、定点飞行等功能}
        % \resumeItem{使用Simscape构建无人机运动仿真模块,配合位置、高度控制器实现运动控制仿真}
        \resumeItem{实现3D环境A*算法,在3D迷宫中为无人机规划路线}
        \resumeItem{应用多项式优化工具实现轨迹规划,成功控制无人机自主起飞、穿越四个随机设置的圆环并降落}
      \resumeItemListEnd
  \resumeSubHeadingListEnd


\section{\href{https://1drv.ms/f/s!Al-YZOpjHxorgslbx6yg1s1e4ZGl1A?e=1DVmBG}{奖项 \faExternalLink}}
    \resumeSubHeadingListStart
    \resumeSubheading
      {2023 Formula Student Germany自动驾驶总冠军}{2023.08}
      {}{}
      \vspace{-4mm}
      
    \resumeSubheading
      {2023 Formula Student East自动驾驶Skidpad第一}{2023.08}
      {}{}
      \vspace{-4mm}

    \resumeSubheading
      {2019钛马大赛 冠军 -(室内环境下搭载单线激光雷达阿克曼转向小车避障越野)}{2019.06}
      {}{}
      \vspace{-4mm}

    \resumeSubheading
      {2018中国工程机器人大赛 冠军 -(工程越野项目竞技赛:摄像头循迹、PID控制、色彩识别、机械臂抓取)}{2018.04}
      {}{}
      \vspace{-4mm}

      % \resumeSubheading
      % {第八届江苏省机器人大赛 冠军 -(机器人越野项目:摄像头循迹、PID控制)}{2017.11}
      % {}{}
      % \vspace{-4mm}

    \resumeSubheading
      {2017中国机器人大赛 一等奖 -(机器人越野项目:摄像头循迹、PID控制)}{2017.08}
      {}{}
      \vspace{-4mm}
      % \resumeSubheading
      %   {National Junior Honor Society}{June 2023 -- Present}
      %   {}{}
      %   \resumeItemListStart
      %   \vspace{-4mm}
      %     \resumeItem{Award presented to students with outstanding achievements in academics, leadership, and community service.}
      %   \resumeItemListEnd
    \resumeSubHeadingListEnd

%-----------Courses-----------
% \section{Classes}
%  \begin{itemize}[leftmargin=0.15in, label={}]
%     \small{\item{
%     \vspace{1mm}
%      \textbf{Courses}{: AP Calulus BC, AP Chemistry, AP Physics, AP World History, AP U.S. History, AP Statistics, AP Literature} \\

%     }}
%  \end{itemize}

%-----------Skill and Interests-----------
\section{技能能力}
 \begin{itemize}[leftmargin=0.15in, label={}]
    \small{\item{
    \vspace{1mm}
     \textbf{编程相关}{: Python、C/C++、Matlab/Simulink、Linux、 ROS、Docker、git} \\
     \vspace{1mm}
     \textbf{专业相关}{: PID/LQR/MPC等控制方法、深度强化学习,无人机/车辆/潜航器的建模和控制} \\
     \vspace{1mm}
     \textbf{语言相关}{:英语雅思6.5(口语7.0),曾深度参与外国工程团队,具有较强沟通能力}

    }}
 \end{itemize}

%-----------Contact-----------
%  \begin{center}
% \small 123-456-7890 $|$
% \href{mailto:ADD EMAIL HERE@x.com}{\underline{andrew@gmail.com}} $|$
% \href{ADD LINKEDIN PAGE HERE}{\underline{linkedin.com/in/andrew}} 
% \end{center}

\end{document}
